%%%%%%%%%%%%%%%%%%%%%%%%%%%%%%%%%%%%%%%%%%%%%%%%%%
%% 京都工芸繊維大学 機械工学課程/機械物理学専攻・機械設計学専攻
%% 卒業論文・修士論文の執筆要綱もとに作成した LaTeX テンプレートです
%% by Motoaki Hiraga (ver.1.0, Oct. 22nd, 2024)
%%%%%%%%%%%%%%%%%%%%%%%%%%%%%%%%%%%%%%%%%%%%%%%%%%


\documentclass[uplatex,dvipdfmx]{kit_mech_thesis}
% \documentclass[lualatex]{kit_mech_thesis}

% 卒論・修論の表紙・背表紙用データの読み込み
%%%%%%%%%%%%%%%%%%%%%%%
%%%%%%% 論文情報記入欄 %%%%%%%
%%%%%%%%%%%%%%%%%%%%%%%

% 卒論なら B, 修論(機械物理学専攻)なら Mp, 修論(機械設計学専攻)なら Md と入力 
\newcommand{\ThesisType}{Mp}

% 令和何年度
\newcommand{\AcademicYear}{00}

% 提出日(英語),卒論の場合は未使用
\newcommand{\DateSubmitted}{February **, 20**}

% 論文題目(日本語), 1行目と2行目を分けて入力(脱字に注意)
\newcommand{\ThesisTitleOneJP}{機械工学課程/機械物理学専攻/機械設計学専攻}
\newcommand{\ThesisTitleTwoJP}{卒論・修論 {\LaTeX} テンプレート}

% 論文題目(英語), 卒論未使用
\newcommand{\ThesisTitleEN}{Unofficial {\LaTeX} Template for a Thesis in\\
    the Department of Mechanical and\\
    System Engineering, the Division of\\
    Mechanophysics/Mechanodesign}
\newcommand{\NumLinesTitleEN}{4} % 英語題目の行数(4行まで)

% 学籍番号
\newcommand{\StudentNum}{00000000}

% 名前(日本語・英語)
\newcommand{\MyNameJP}{平賀 元彰}
\newcommand{\MyNameEN}{Motoaki HIRAGA} % 卒論未使用

% 主任指導教員の名前(日本語・英語)
\newcommand{\ChiefAdvisorJP}{熱流 一郎}
\newcommand{\ChiefAdvisorTitleJP}{教 授}
\newcommand{\ChiefAdvisorEN}{Professor Ichiro NETSURYU} % 卒論未使用

% 主任指導教員と指導教員の両方を記載する場合は True, 卒論未使用
\newcommand{\MultiAdvisors}{True}

% 指導教員の名前(日本語・英語),卒論未使用
\newcommand{\AdvisorJP}{材強 次郎}
\newcommand{\AdvisorTitleJP}{講 師}
\newcommand{\AdvisorEN}{Lecturer Jiro ZAIKYOU}


%%%%%%% 表紙・背表紙の印刷部数 %%%%%%%
\newcommand{\TitleCopyNumber}{4}

% 背表紙用のタイトル
\newcommand{\SpineTitle}{\ThesisTitleOneJP\ThesisTitleTwoJP}
% \newcommand{\SpineTitle}{タイトルが長いなどの理由で任意の位置に改行を\newline
%                          入れたい場合はこの例のよう入力(誤字・脱字に注意)}


% 縦書き用パッケージの読み込み
\ifluatex
    % LuaTeX の場合
    \usepackage{lltjext}
    \usepackage{ltj-latex}
    \newcommand{\kanjiskip}{\ltjsetkanjiskip}
\else
    % その他の場合(pLaTeX/upLaTeXのみ対応)
    \usepackage{plext}
\fi

% 背表紙の整列に使用
\usepackage{array}

% 出力回数のカウンタ
\newcounter{CoverCopyCount}
\newcounter{SpineCopyCount}


%%%%% ドキュメントの開始 %%%%%
\begin{document}

\frontmatter

%%%%% 表紙 %%%%%
\setcounter{CoverCopyCount}{\TitleCopyNumber}
\whiledo{\value{CoverCopyCount}>0}{

    %%%%%%%%%%%%%%%%%%%%%%%%%%%%%%%%%%%%%%%%%%%%%%%%%%
%% 京都工芸繊維大学 機械工学課程/機械物理学専攻・機械設計学専攻
%% 卒業論文・修士論文の執筆要綱もとに作成した LaTeX テンプレートです
%% by Motoaki Hiraga (ver.1.0, Oct. 22nd, 2024)
%%%%%%%%%%%%%%%%%%%%%%%%%%%%%%%%%%%%%%%%%%%%%%%%%%

%%%%%%%%%%%%%%%%%%%%%%%%%%%%%%%%%%%%%%%%
%%%%%%%%%% 表紙および中表紙用 TeX ファイル %%%%%%%%%%
% src/thesis_data.tex の「論文情報記入欄」に記入してください.
%%%%%%%%%%%%%%%%%%%%%%%%%%%%%%%%%%%%%%%%

%%%%%%%%%%%%%%%%%%%%%%%%%%%%%%%%%%%%%%%%
%%%%%%%%%% 表紙および中表紙に関する注意(要約) %%%%%%%%%%
% ・卒論は日本語版のみ,修論は日本語版と英語版の両方を含める.
% ・修論の場合,指導の形態により,主任指導教員と指導教員の
%  両方を記載する場合と,主任指導教員のみを記載する場合がある.
% ・英語で執筆した場合も,日本語版の表紙および中表紙を用いる.
%  このとき,可能な限り,題目以外を日本語で記載する.
% ・PDF には,中表紙と本文すべてを含める.
%%%%%%%%%%%%%%%%%%%%%%%%%%%%%%%%%%%%%%%%

%%%% 日本語版の表紙および中表紙
\begin{titlepage}
  \titlepagestyle
  \begin{center}

    \ifthenelse{{\equal{\ThesisType}{B}}\OR{\equal{\ThesisType}{Mp}}\OR{\equal{\ThesisType}{Md}}}
    {}{{\Huge {\ThesisTypeError}}}

    \vspace*{4.6truemm}
    {\fontsize{18pt}{18pt}\selectfont 令和 {\AcademicYear} 年度}

    \vspace{14truemm}
    \ifthenelse{\equal{\ThesisType}{B}}{
      {\fontsize{26.5pt}{26.5pt}\selectfont
          \contour{black}{卒\hspace{10.7truemm}業\hspace{10.7truemm}論\hspace{10.7truemm}文}}
    }{
      {\fontsize{26.5pt}{26.5pt}\selectfont
          \contour{black}{修\hspace{10.7truemm}士\hspace{10.7truemm}論\hspace{10.7truemm}文}}
    }

    \vspace{13.7truemm}
    {\fontsize{18pt}{18pt}\selectfont \textrm{題\hspace{12.5truemm}目}}

    \vspace{18truemm}
    {\fontsize{22pt}{22pt}\selectfont \textbf{\ThesisTitleOneJP}}

    \vspace{-2.5truemm}
    \noindent
    \rule{123truemm}{0.7pt}

    \vspace{14.8truemm}
    {\fontsize{22pt}{22pt}\selectfont \textbf{\ThesisTitleTwoJP}}

    \vspace{-2.5truemm}
    \noindent
    \rule{123truemm}{0.7pt}

    \vspace{25truemm}
    \ifthenelse{{\equal{\ThesisType}{B}}}{
      % 卒論
      \begin{tabbing}
        % 左余白 \> 項目 \> スペース \> 下線開始 \> 記入情報 \\[改行時行間]
        \hspace{0.5truemm} \= \hspace{26truemm} \= \hspace{-0.5truemm} \= \hspace{11truemm} \= \kill
        \> \makebox[26truemm]{\fontsize{16pt}{16pt}\selectfont 学{\hfill}籍{\hfill}番{\hfill}号} \> \> \> {\fontsize{20pt}{20pt}\selectfont \textsf{\StudentNum}}  \\[-5.2truemm]
        \>  \>  \> \rule{93truemm}{0.7pt} \>  \\[3.65truemm]
        \> \makebox[26truemm]{\fontsize{16pt}{16pt}\selectfont 提{\hfill}出{\hfill}者} \> \> \> {\fontsize{20pt}{20pt}\selectfont \textsf{\MyNameJP}}  \\[-5.2truemm]
        \>  \>  \> \rule{93truemm}{0.7pt} \>  \\[16.3truemm]
        \> \makebox[26truemm]{\fontsize{16pt}{16pt}\selectfont 指{\hfill}導{\hfill}教{\hfill}員} \> \> \> {\fontsize{20pt}{20pt}\selectfont \textsf{{\ChiefAdvisorJP}\hspace{10truemm}{\ChiefAdvisorTitleJP}}}  \\[-5.2truemm]
        \>  \>  \> \rule{93truemm}{0.7pt} \>  \\[3.65truemm]
      \end{tabbing}
    }{
      % 修論
      \begin{tabbing}
        % 左余白 \> 項目 \> スペース \> 下線開始 \> 記入情報 \\[改行時行間]
        \hspace{0.5truemm} \= \hspace{26truemm} \= \hspace{-0.5truemm} \= \hspace{11truemm} \= \kill
        \> \makebox[26truemm]{\fontsize{16pt}{16pt}\selectfont 申{\hfill}請{\hfill}者} \> \> \> {\fontsize{20pt}{20pt}\selectfont \textsf{\MyNameJP}\textsf{({\StudentNum})}}  \\[-5.2truemm]
        \>  \>  \> \rule{93truemm}{0.7pt} \>  \\[16.3truemm]
        \> \makebox[26truemm]{\fontsize{14pt}{14pt}\selectfont 主{\hfill}\!任{\hfill}\!指{\hfill}\!導{\hfill}\!教{\hfill}\!員} \> \> \> {\fontsize{20pt}{20pt}\selectfont \textsf{{\ChiefAdvisorJP}\hspace{10truemm}{\ChiefAdvisorTitleJP}}}  \\[-5.2truemm]
        \ifthenelse{{\equal{\MultiAdvisors}{True}}}{
        % 主任指導教員と指導教員が異なる場合
        \>  \>  \> \rule{93truemm}{0.7pt} \>  \\[3.65truemm]
        \> \makebox[26truemm]{\fontsize{16pt}{16pt}\selectfont 指{\hfill}導{\hfill}教{\hfill}員} \> \> \> {\fontsize{20pt}{20pt}\selectfont \textsf{{\AdvisorJP}\hspace{10truemm}{\AdvisorTitleJP}}}  \\[-5.2truemm]
        \>  \>  \> \rule{93truemm}{0.7pt} \>  \\[3.65truemm]
        }{
        \>  \>  \> \rule{93truemm}{0.7pt} \>  \\[16.3truemm]
        }

      \end{tabbing}
    }

    \vspace{3.8truemm}
    \ifthenelse{{\equal{\ThesisType}{B}}}{
      % 卒論
      {\fontsize{16pt}{16pt}\selectfont 京\,都\,工\,芸\,繊\,維\,大\,学\hspace{4.8truemm}工\,芸\,科\,学\,部}

      \vspace{6.3truemm}
      {\fontsize{16pt}{16pt}\selectfont 機\,械\,工\,学\,課\,程}
    }{
      % 修論
      {\fontsize{16pt}{16pt}\selectfont 京\,都\,工\,芸\,繊\,維\,大\,学\hspace{7.8truemm}大\,学\,院\,工\,芸\,科\,学\,研\,究\,科}

      \vspace{6.3truemm}
      \ifthenelse{{\equal{\ThesisType}{Mp}}}{
        % 機械物理学専攻
        {\fontsize{16pt}{16pt}\selectfont 機\,械\,物\,理\,学\,専\,攻}
      }{
        % 機械設計学専攻
        {\fontsize{16pt}{16pt}\selectfont 機\,械\,設\,計\,学\,専\,攻}
      }
    }
  \end{center}

  \clearpage

  %%%% 英語版の表紙および中表紙(修論のみ)
  \ifthenelse{{\equal{\ThesisType}{B}}\OR\NOT{\isundefined{\TitleCopyNumber}}}{}{
    \begin{center}

      \vspace*{15.85truemm}
      \parbox[t][20.5truemm][t]{\textwidth}{
        \begingroup
        \begin{center}
          \renewcommand{\\}{\newline}
          \setlength{\baselineskip}{12.5truemm}
          {\fontsize{23pt}{23pt}\selectfont \textbf{\ThesisTitleEN}}
        \end{center}
        \endgroup
      }

      \vspace{12truemm * (\NumLinesTitleEN - 1)}
      {\fontsize{16pt}{16pt}\selectfont by}

      \vspace{11truemm}
      {\fontsize{20pt}{20pt}\selectfont {\MyNameEN}}

      \vspace{8.2truemm}
      {\fontsize{15.5pt}{24.1pt}\selectfont
        A thesis submitted\\
        to\\
        \ifthenelse{{\equal{\ThesisType}{Mp}}}{
          % 機械物理学専攻
          Division of Mechanophysics,\\
        }{
          % 機械設計学専攻
          Division of Mechanodesign,\\
        }
        Graduate School of Science and Technology\\
        in partial fulfillment of the requirements\\
        for the degree\\
        of\\[2.3truemm]
        Master of Engineering
      }

      \vspace{8.5truemm}
      {\fontsize{17.5pt}{24pt}\selectfont Chief Advisor:~{\ChiefAdvisorEN}\\
      \ifthenelse{{\equal{\MultiAdvisors}{True}}}{
      % 主任指導教員と指導教員が異なる場合
      Advisor:~{\AdvisorEN}\\
      }{
      \,\\
      }
      }

      \vspace{11.45truemm}
      {\fontsize{15pt}{15pt}\selectfont Kyoto Institute of Technology\\[2.15truemm]
        Sakyo, Kyoto}

      \vspace{10.5truemm}
      {\fontsize{15pt}{15pt}\selectfont {\DateSubmitted}}

    \end{center}
  }
  \cleartitlepagestyle
\end{titlepage}

\clearpage

    \addtocounter{CoverCopyCount}{-1}
}

%%%%% 背表紙 %%%%%
\setcounter{SpineCopyCount}{\TitleCopyNumber}
%%%%%%%%%%%%%%%%%%%%%%%%%%%%%%%%%%%%%%%%%%%%%%%%%%
%% 京都工芸繊維大学 機械工学課程/機械物理学専攻・機械設計学専攻
%% 卒業論文・修士論文の執筆要綱もとに作成した LaTeX テンプレートです
%% by Motoaki Hiraga (ver.1.0, Oct. 22nd, 2024)
%%%%%%%%%%%%%%%%%%%%%%%%%%%%%%%%%%%%%%%%%%%%%%%%%%

%%%%%%%%%%%%%%%%%%%%%%%%%%%%%%%%%%%%%%%%
%%%%%%%%%% 背表紙用 TeX ファイル %%%%%%%%%%
%%%%%%%%%%%%%%%%%%%%%%%%%%%%%%%%%%%%%%%%

\whiledo{\value{SpineCopyCount}>0}{
    % タイトルの幅 & スペース & 年度の幅 & スペース & 名前の幅
    \begin{tabular*}<t>{\textheight}{m{0.61\textheight}m{0.0\textheight}m{0.14\textheight}m{0.0\textheight}m{0.22\textheight}}
        {\fontsize{18pt}{18pt}\selectfont \textsf{\SpineTitle}} & &\setlength{\kanjiskip}{4pt} {\fontsize{14pt}{14pt}\selectfont \textsf{令和 \rensuji{\AcademicYear} 年度}} && \setlength{\kanjiskip}{4pt} {\fontsize{18pt}{18pt}\selectfont \textsf{\MyNameJP}}\\
    \end{tabular*}
    \addtocounter{SpineCopyCount}{-1}
}



\end{document}